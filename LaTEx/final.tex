\documentclass{article}


\usepackage[utf8]{inputenc}% phông chữ
\usepackage[T5]{fontenc} %sử dụng tiếng việt
\usepackage[fontsize=13pt]{scrextend} % set font
\usepackage[paperheight=29.7cm, paperwidth=21cm, right=2cm, 
left=3cm, top=2cm,bottom = 2.5cm]{geometry} % chuan A4, can le phai, trai, tren, duoi
\usepackage{mathptmx} %time new roman
\usepackage{graphicx} % chen anh
\usepackage{float} % set vi tri chen anh
\usepackage{tikz} % thu vien tao khung bia
\usetikzlibrary{calc} %thu vien tikz
\usepackage{indentfirst} % thư viện thụt đầu dòng
\renewcommand{\baselinestretch}{1.2} % dãn dòng 1.2
\setlength{\parskip}{6pt} % spacing after
\setlength{\parindent}{1cm} % xét khoảng cách thụt đầu dòng mỗi đoạn
\usepackage{titlesec} % thư viện setup font chữ
\setcounter{secnumdepth}{4} % có 4 heading
\titlespacing*{\section}{0pt}{0pt}{30pt}
\titleformat*{\section}{\fontsize{16pt}{0pt}\selectfont \bfseries \centering}

\titlespacing*{\subsection}{0pt}{10pt}{0pt} %heading 2
\titleformat*{\subsection}{\fontsize{14pt}{0pt}\selectfont \bfseries}

\titlespacing*{\subsubsection}{0pt}{10pt}{0pt} %heading 3
\titleformat*{\subsubsection}{\fontsize{13pt}{0pt}\selectfont \bfseries \itshape}

\titlespacing*{\paragraph}{0pt}{10pt}{0pt} %heading 4
\titleformat*{\paragraph}{\fontsize{13pt}{0pt}\selectfont \itshape}

% thiết lập phần mô tả dưới hình được trích dẫn
\renewcommand{\figurename}{\fontsize{12pt}{0pt}\selectfont \bfseries Hình}
\renewcommand{\thefigure}{\thesection.\arabic{figure}} % được ghi bằng phần.1, .2,...
\usepackage{caption}
\captionsetup[figure]{labelsep= space} % bỏ dấu ":". từ "Hình 1.1:" chuyển thành "Hình 1.1 "

% thiết lập phần mô tả trên bảng được trích dẫn
\renewcommand{\tablename}{\fontsize{12pt}{0pt}\selectfont \bfseries Bảng}
\renewcommand{\thetable}{\thesection.\arabic{table}} % được ghi bằng phần.1, .2,...
\captionsetup[table]{labelsep= space} % bỏ dấu ":". từ "Hình 1.1:" chuyển thành "Hình 1.1 "
\usepackage{tabularx}
\newcolumntype{s}{>{\hsize= 0.4\hsize}X}
\newcolumntype{a}{>{\hsize= 1.1\hsize}X}

\renewcommand{\theequation}{\thesection.\arabic{equation}} % thay đổi đánh số phương trình mặc định

\newtheorem{theorem}{Định lý}
\newtheorem{defn}[theorem]{Định nghĩa}
\newtheorem{corollary}[theorem]{Hệ quả}
\newtheorem{lemma}[theorem]{Bổ đề}

\usepackage{lipsum} % THƯ viện tạo chữ linh tinh

\renewcommand{\contentsname}{MỤC LỤC}

\usepackage[unicode]{hyperref}% nhảy đến địa chỉ khi nhấn vào mục lục




\begin{document}

\begin{titlepage}

\begin{tikzpicture}[overlay, remember picture]
\draw[line width =3pt]
    ($ (current page.north west)+(3.0cm, -2.0cm) $)
    rectangle
    ($ (current page.south east)+( -2.0cm, 2.5cm)$);
\draw[line width =0.5pt]
    ($(current page.north west)+(3.1cm, -2.1cm)$)
    rectangle
    ($(current page.south east)+( -2.1cm, 2.6cm)$);
\end{tikzpicture}

\begin{center}
    \vspace{-6pt} ĐẠI HỌC BÁCH KHOA HÀ NỘI \\
    \textbf{\fontsize{16pt}{0pt}\selectfont TRƯỜNG CÔNG NGHỆ THÔNG TIN VÀ TRUYỀN THÔNG}
    \vspace{0.5cm}
    \begin{figure}[H]
        \centering
        \includegraphics[width=1.5cm, height=2.26cm]{images/bk_logo.png}
    \end{figure}
\vspace{0.5cm}
\fontsize{38pt}{0pt}\selectfont MINI PROJECT\\
\vspace{12pt}
\fontsize{20pt}{0pt}\selectfont HỌC PHẦN: NHẬP MÔN KỸ THUẬT \\
\fontsize{20pt}{0pt}\selectfont TRUYỀN THÔNG \\
\vspace{0.5cm}
\textit{\fontsize{20pt}{0pt}\selectfont (MÃ HỌC PHẦN: IT4593)}
\end{center}


\vspace{6pt}
\textbf{\fontsize{16pt}{0pt}\selectfont Đề tài:}

\begin{center}
\textbf{\fontsize{20pt}{0pt}\selectfont ĐIỀU CHẾ VÀ GIẢI ĐIỀU CHẾ TÍN HIỆU }\\
\textbf{\fontsize{20pt}{0pt}\selectfont KHÓA DỊCH PHA (PHASE SHIFT KEYING PSK)}\\
\textbf{\fontsize{20pt}{0pt}\selectfont  TỪ MỘT CHUỖI NHỊ PHÂN NGẪU NHIÊN.}\\
\end{center}


\vspace{1cm}
\begin{table}[H]
    \centering
    \begin{tabular}{1 1}
    \fontsize{14pt}{0pt}\selectfont Sinh viên thực hiện: &  \fontsize{20pt}{0pt}\selectfont Nguyễn Ánh Vân \vspace{10pt}\\
   \fontsize{14pt}{0pt}\selectfont Mã số sinh viên:  & \fontsize{20pt}{0pt}\selectfont 20205045 \vspace{10pt}\\ 
   \fontsize{14pt}{0pt}\selectfont Lớp:  & \fontsize{20pt}{0pt}\selectfont 135279 \vspace{10pt}\\
   \fontsize{14pt}{0pt}\selectfont Giảng viên hướng dẫn:  & \fontsize{20pt}{0pt}\selectfont TS. Trịnh Văn Chiến \\
\end{tabular}
\end{table}

\begin{center}
    \vspace{1.75cm}
\fontsize{14pt}{0pt}\selectfont Hà Nội, Tháng 1 năm 2023
\end{center}
\end{titlepage}
\cleardoublepage % trang khác
\thispagestyle{empty} % khong danh so trang

\begin{titlepage}

\begin{tikzpicture}[overlay, remember picture]
\draw[line width =3pt]
    ($ (current page.north west)+(3.0cm, -2.0cm) $)
    rectangle
    ($ (current page.south east)+( -2.0cm, 2.5cm)$);
\draw[line width =0.5pt]
    ($(current page.north west)+(3.1cm, -2.1cm)$)
    rectangle
    ($(current page.south east)+( -2.1cm, 2.6cm)$);
\end{tikzpicture}

\begin{center}
    \vspace{-6pt} ĐẠI HỌC BÁCH KHOA HÀ NỘI \\
    \textbf{\fontsize{16pt}{0pt}\selectfont TRƯỜNG CÔNG NGHỆ THÔNG TIN VÀ TRUYỀN THÔNG}
    \vspace{0.5cm}
    \begin{figure}[H]
        \centering
        \includegraphics[width=1.5cm, height=2.26cm]{images/bk_logo.png}
    \end{figure}
\vspace{0.5cm}
\fontsize{38pt}{0pt}\selectfont MINI PROJECT\\
\vspace{12pt}
\fontsize{20pt}{0pt}\selectfont HỌC PHẦN: NHẬP MÔN KỸ THUẬT \\
\fontsize{20pt}{0pt}\selectfont TRUYỀN THÔNG \\
\vspace{0.5cm}
\textit{\fontsize{20pt}{0pt}\selectfont (MÃ HỌC PHẦN: IT4593)}
\end{center}


\vspace{6pt}
\textbf{\fontsize{16pt}{0pt}\selectfont Đề tài:}

\begin{center}
\textbf{\fontsize{20pt}{0pt}\selectfont ĐIỀU CHẾ VÀ GIẢI ĐIỀU CHẾ TÍN HIỆU }\\
\textbf{\fontsize{20pt}{0pt}\selectfont KHÓA DỊCH PHA (PHASE SHIFT KEYING PSK)}\\
\textbf{\fontsize{20pt}{0pt}\selectfont  TỪ MỘT CHUỖI NHỊ PHÂN NGẪU NHIÊN.}\\
\end{center}


\vspace{1cm}
\begin{table}[H]
    \centering
    \begin{tabular}{1 1}
    \fontsize{14pt}{0pt}\selectfont Sinh viên thực hiện: &  \fontsize{20pt}{0pt}\selectfont Nguyễn Ánh Vân \vspace{10pt}\\
   \fontsize{14pt}{0pt}\selectfont Mã số sinh viên:  & \fontsize{20pt}{0pt}\selectfont 20205045 \vspace{10pt}\\ 
   \fontsize{14pt}{0pt}\selectfont Lớp:  & \fontsize{20pt}{0pt}\selectfont 135279 \vspace{10pt}\\
   \fontsize{14pt}{0pt}\selectfont Đội bảo vệ:  & \fontsize{20pt}{0pt}\selectfont blabla \\
\end{tabular}
\end{table}

\begin{center}
    \vspace{0.5cm}
    \fontsize{20pt}{0pt}\selectfont trong đồ án cần trang thứ 2 gồm có những thành viên bảo vệ đồ án \\
    \vspace{0.5cm}
\fontsize{14pt}{0pt}\selectfont Hà Nội, Tháng 1 năm 2023
\end{center}
\end{titlepage}
\cleardoublepage

\section*{LỜI NÓI ĐẦU} % dấu * để không đánh dấu trang
\thispagestyle{empty}
Phàn này trình bày một cách khái quát(khoảng 1-2 trang) về bối cảnh hình thành và mục đích của đồ án, project. Lời cảm ơn với những tổ chức và cá nhân góp phần trong việc hoàn thiện đồ án nên đặt ở cuối mục này.

\cleardoublepage
\section*{LỜI CAM ĐOAN} % dấu * để không đánh dấu trang
\thispagestyle{empty}
TÔI tên là Nguyễn Ánh Vân, mssv 20205045, sinh viên lớp, khóa, người hướng dẫn là. tôi xin cam đoan toàn bộ nội dung được trình bày trong đồ án *tên* là kết quả quá trình tìm hiểu và nghiên cứu của tôi. các dữ liệu được nêu trong đồ án là hoàn toàn trung thực, phản ánh kết quả đo đạt thực tế. mọi thông tin trích dẫn đều tuân thủ các quy định về sở hữu trí tuệ: các tài liệu tham khảo được liệt kê rõ ràng. tôi xin chịu hoàn toàn trách nhiệm với những nội dung được viết trong đồ án này.

\vspace{6pt}
\hspace{7cm}
Hà Nội, ngày 30 tháng 1 năm 2023

\hspace{9cm}\textbf{Người cam đoan} 
\vspace{2cm}

\hspace{8.65cm} \textbf{NGUYỄN ÁNH VÂN}
\cleardoublepage

\addtocontents{toc}{\protect\thispagestyle{empty}}
\tableofcontents % tạo mục lục tự động
\thispagestyle{empty}
\cleardoublepage

\pagenumbering{roman} %số trang bằng chữ la mã
\section*{DANH MỤC KÝ HIỆU VÀ CHỮ VIẾT TẮT}
\phantomsection
\addcontentsline{toc}{section}{\numberline{} DANH MỤC KÝ HIỆU VÀ CHỮ VIẾT TẮT}
\cleardoublepage

{
\let\oldnumberline\numberline
\renewcommand{\numberline}{\figurename~\oldnumberline}
\listoffigures} % tạo danh mục hình tự động
\phantomsection
\addcontentsline{toc}{section}{\numberline{} DANH MỤC HÌNH vẼ}
\cleardoublepage

{
\let\oldnumberline\numberline
\renewcommand{\numberline}{\tablename~\oldnumberline}
\listoftables
}
\phantomsection
 % tạo danh mục bảng biểu tự động
\addcontentsline{toc}{section}{\numberline{} DANH MỤC BẢNG BIỂU}
\cleardoublepage

\section*{TÓM TẮT ĐỒ ÁN}
\phantomsection
\addcontentsline{toc}{section}{\numberline{} TÓM TẮT ĐỒ ÁN}
phần này trình bày những mục đích, các kết luận quan trọng nhất của đồ án bằng cả tiếng việt và tiếng anh.
\cleardoublepage

\pagenumbering{arabic} % số trang bằng số thứ tự 1,2,3,...
\section*{CHƯƠNG 1. CHƯƠNG MỞ DẦU}
\phantomsection
\addcontentsline{toc}{section}{\numberline{} CHƯƠNG 1: CHƯƠNG MỞ DẦU}
\setcounter{subsection}{0}
\setcounter{figure}{0}
\setcounter{table}{0}
phần mở đầu goiwis thiệu vấn đề mà đồ án cần giải quyết, mô tả được các phương pháp hiện có để giải quyết vấn đề, trình bày mục đích của đồ án song song với việc giới hạn phạm vi của vấn đề mà đồ án sẽ cần phải giải quyết. phần này sẽ giới thiệu tóm tắt cấu trúc đồ án, nội dung tương ứng của các phần sẽ lần lượt được trình bày ở các chương tiếp theo.
nội dung chính của đồ án tốt nghiệp thường bao gồm: 
\begin{itemize} % tạo list, phần hiển thị có chấm tròn, dạng liệt kê
    \item Phần mở đầu giới thiệu đề tài
    \item Một chương giới thiệu cơ sở lý thuyết
    \item Một hoặc nhiều chương trình bày các vấn đề 
    \item Một chương mô tả các thí nghiệm và kết quả thu được
\end{itemize}
\cleardoublepage
\newpage

\section*{CHƯƠNG 2. CƠ SỞ LÝ THUYẾT}
\phantomsection
\addcontentsline{toc}{section}{\numberline{} CHƯƠNG 2. CƠ SỞ LÝ THUYẾT}
\setcounter{section}{2}
\setcounter{subsection}{0}
\setcounter{figure}{0}
\setcounter{table}{0}
Mỗi chương sẽ bắt đầu bằng một đoạn giới thiệu các phần chính được trình bày trong đó từ 5 đến 10 dòng và kết thúc bằng một đoạn tóm tắt các kết luận chính của chương. chiều dài của chương cho cân đối và hợp lý.

\subsection{Một số lưu ý khi trình bày đồ án}
Sau đây là một vài chú ý khi làm đồ án tốt nghiệp:
\subsubsection{Nộp đồ án}
sinh viên (nhóm tối đa 3 người) làm một đề tài nộp 2 quyển đồ án tốt nghiệp tại văn phòng bộ môn của giảng viên hướng dẫn trước ngày bảo vệ ít nhất 1 tuần. Một quyển đồ án cần:
\begin{itemize}
    \item được \textbf{in 2 mặt} nhằm tiết kiệm không gian lưu trữ
    \item đóng bìa mềm bên ngaoif và bóng kính
    \item số trang: 50-150trang, không keer phần phụ lục
    \item phải có chữ ký sinh viên sau \textbf{lời cam đoan và của giảng viên hướng dẫn}
\end{itemize}

\subsubsection{Phụ lục}
Phụ lục (nếu có)) chứa các thông tin liên quan của đồ án nhưng nếu để owr trong phần chính sẽ gây rườm rà. Thông thường các chi tiết được để ở phần phụ lục là kết quả thô(chưa qua xử lý), mã nguồn, phần mềm, thư viện, thông số kỹ thuật, chi tiết linh kiện, hình ảnh minh họa thêm,...
\subsubsection{Tài liệu tham khảo}
\paragraph{Cách liệt kê} \mbox{}

áp dụng cách liệt kê theo quy định của IEEE. theo đó, tài liệu tham khảo được đánh số thứ tự trong ngoặc vuông. thứ tự liệt kê là thứ tự xuất hiện của tài liệu tham khảo được trích dẫn. tài liệu tham khao đã liệt kê bắt buộc phải được trích dẫn trong phần nội dung của đồ án. tài liệu tham khảo cần có nguồn gốc rõ ràng và phải từ nguồn tin cậy. hạn chế trích dẫn tài liệu tham khảo từ các website, wikimedia
\paragraph{Các loại tài liệu tham khảo}\mbox{}

các nguồn tài liệu tham khảo chính là sách, bài báo trong các tạp chí, bài báo trong các hội nghị và các tài liệu tham khảo khác trên internet. 

\subsubsection{Đánh số phương trình}
phương trình được đánh số theo số của chương. như hình sẽ và bảng biểu (phương trình trong chương 1 là 1.1, 1.2,...

\subsubsection{Đánh số định nghĩa, định lý, kết quả}
các định nghĩa, định lý, kết quả được đánh số theo số của chương và được sử dụng chung một chỉ số. ví dụ trong chương 3, các định nghĩa, định lý, hệ quả sẽ được đánh số theo thứ tự như sau: định lý 3.1, định nghĩa 3.2, hệ quả 3.3,...

\newpage
\section*{CHƯƠNG 3. THUẬT TOÁN}
\phantomsection
\addcontentsline{toc}{section}{\numberline{} CHƯƠNG 3. THUẬT TOÁN}
\setcounter{section}{3}
\setcounter{subsection}{0}
\setcounter{figure}{0}
\setcounter{table}{0}
Đây là phần sinh viên tự phát triển như:
\begin{itemize}
    \item xây dựng thuật toán
    \item xây dựng chương trình
    \item mô phỏng, tính toán, thiết kế, chạy thử kết quả
\end{itemize}
\subsection{Cách chèn ẳnh}
\begin{figure}[H]
    \centering
    \includegraphics[width=10cm, height= 10cm]{images/bk_logo.png}
    \caption[Logo Hust]{\fontsize{12pt}{0pt}\selectfont \bfseries{Logo của Đại học Bách Khoa Hà Nội}}
    \label{hình31}
\end{figure}
Hình \ref{hình31} là ví dụ về cách chèn ảnh. lưu ý chú thích của hình vẽ được đặt ngay dưới hình vẽ. tất cả các hình vẽ phải được đề cập đến trong phần nội dung và phải được phân tích và bình luận giống như đang làm :3

\subsection{Cách tạo bảng}
\begin{table}[H]
    \centering
    \caption{Kết quả thí nghiệm}
    \label{bang31}
    \begin{tabularx}{0.85\textwidth}{|>{\centering\arraybackslash}s|>{\centering\arraybackslash}a|>{\centering\arraybackslash}a|>{\centering\arraybackslash}s|}
    \hline
      Lần thí nghiệm   & điện áp đo được  & điện áp tham chiếu  & sai lệch(\%) \\
      \hline
        1 &  &  &\\
         \hline
        2 &  &  &\\
         \hline
        3 &  &  &\\
         \hline
    \end{tabularx}
\end{table}
bảng \ref{bang31} là ví dụ về cách tạo bảng. lưu ý chú thích của bảng được đặt ở ngay trước của bảng. tất cả các bảng biểu phải được đề cập đến trong phần nội dung và phải được phân tích và bình luận giống như nè

\subsection{Cách viết phương trình}
\begin{equation} \label{pt31}
    F(x) = \int^a_b \frac{1}{3}x^3
\end{equation}
Phương trình \ref{pt31} là một phương trình tích phân, là ví dụ về 1 phương tình tích phân. 

Thử phương trình khác
\begin{equation} \label{pt32}
    F(x) = \frac{a}{b}
\end{equation}
Phương trình \ref{pt32} là một phân số dạng thông thường.

\subsection{Cách viết định nghĩa, hẹ quả,...}
Định lý Py-ta-go được áp dụng trong tam giác vuông.
\begin{theorem}\label{dlptg} % định lý
Định lý Py-ta-go: trong một tam giác vuông ta luôn có bình phương của cạnh huyền sẽ bằng tổng bình phương của hai cạnh góc vuông.
\end{theorem}
định lý \ref{dlptg} được áp dụng trong tam giác vuông

\begin{corollary}\label{corol}
blabla

\end{corollary}
Hệ quả \ref{corol} blabla

\begin{lemma}\label{lemma}
hihi bổ đề nha
\end{lemma}
bổ đề \ref{lemma} hihi

\begin{defn}\label{defn}
nội dung :3
\end{defn}
định nghĩa \ref{defn} :3
\newpage

\section*{CHƯƠNG 4. THÍ NGHIỆM VÀ KẾT QUẢ}
\phantomsection
\addcontentsline{toc}{section}{\numberline{} CHƯƠNG 4. THÍ NGHIỆM VÀ KẾT QUẢ}
\setcounter{section}{4}
\setcounter{subsection}{0}
\setcounter{figure}{0}
\setcounter{table}{0}
\lipsum[] 
\begin{figure}[H]
    \centering
    \includegraphics[width=10cm, height=12cm]{images/bk_logo.png}
    \label{Hinh41}
    \caption{Hinh Bách khoa}
\end{figure}
Hình \ref{Hinh41} hihi
\cleardoublepage

\section*{KẾT LUẬN}
\phantomsection
\addcontentsline{toc}{section}{\numberline{} KẾT LUẬN}
\subsection*{Kết luận chung}
\addcontentsline{toc}{section}{\numberline{}Kết luận chung}
kết luận chung cho các chương trong đồ án. Mục này cần nhấn mạnh những vấn đề đã giải quyết và chưa giải quyết để đưa ra các đánh giá về mức độ hoàn thành công việc. thường bao gồm việc so sánh kết quả thu được với mục tiêu đề ra ban đầu.
\subsection*{Hướng phát triển}
\addcontentsline{toc}{section}{\numberline{}Hướng phát triển}
\subsection*{Kiến nghị và đề xuất}
\addcontentsline{toc}{section}{\numberline{}Kiến nghị và đề xuất}
\cleardoublepage

\section*{TÀI LIỆU THAM KHẢO}
\phantomsection
\addcontentsline{toc}{section}{\numberline{}TÀI LIỆU THAM KHẢO}
\bibliography{IEEEtran}
\bibliography{taiLieuThamKhao}
\cleardoublepage

\section*{PHỤ LỤC}
\phantomsection

\end{document}
