\documentclass{article}

\begin{document}

\pagenumbering{arabic} % số trang bằng số thứ tự 1,2,3,...
\section*{CHƯƠNG 1. CHƯƠNG MỞ DẦU}
\phantomsection
\addcontentsline{toc}{section}{\numberline{} CHƯƠNG 1: CHƯƠNG MỞ DẦU}
\setcounter{subsection}{0}
\setcounter{figure}{0}
\setcounter{table}{0}
phần mở đầu goiwis thiệu vấn đề mà đồ án cần giải quyết, mô tả được các phương pháp hiện có để giải quyết vấn đề, trình bày mục đích của đồ án song song với việc giới hạn phạm vi của vấn đề mà đồ án sẽ cần phải giải quyết. phần này sẽ giới thiệu tóm tắt cấu trúc đồ án, nội dung tương ứng của các phần sẽ lần lượt được trình bày ở các chương tiếp theo.
nội dung chính của đồ án tốt nghiệp thường bao gồm: 
\begin{itemize} % tạo list, phần hiển thị có chấm tròn, dạng liệt kê
    \item Phần mở đầu giới thiệu đề tài
    \item Một chương giới thiệu cơ sở lý thuyết
    \item Một hoặc nhiều chương trình bày các vấn đề 
    \item Một chương mô tả các thí nghiệm và kết quả thu được
\end{itemize}
\cleardoublepage
\newpage

\section*{CHƯƠNG 2. CƠ SỞ LÝ THUYẾT}
\phantomsection
\addcontentsline{toc}{section}{\numberline{} CHƯƠNG 2. CƠ SỞ LÝ THUYẾT}
\setcounter{section}{2}
\setcounter{subsection}{0}
\setcounter{figure}{0}
\setcounter{table}{0}
Mỗi chương sẽ bắt đầu bằng một đoạn giới thiệu các phần chính được trình bày trong đó từ 5 đến 10 dòng và kết thúc bằng một đoạn tóm tắt các kết luận chính của chương. chiều dài của chương cho cân đối và hợp lý.

\subsection{Một số lưu ý khi trình bày đồ án}
Sau đây là một vài chú ý khi làm đồ án tốt nghiệp:
\subsubsection{Nộp đồ án}
sinh viên (nhóm tối đa 3 người) làm một đề tài nộp 2 quyển đồ án tốt nghiệp tại văn phòng bộ môn của giảng viên hướng dẫn trước ngày bảo vệ ít nhất 1 tuần. Một quyển đồ án cần:
\begin{itemize}
    \item được \textbf{in 2 mặt} nhằm tiết kiệm không gian lưu trữ
    \item đóng bìa mềm bên ngaoif và bóng kính
    \item số trang: 50-150trang, không keer phần phụ lục
    \item phải có chữ ký sinh viên sau \textbf{lời cam đoan và của giảng viên hướng dẫn}
\end{itemize}

\subsubsection{Phụ lục}
Phụ lục (nếu có)) chứa các thông tin liên quan của đồ án nhưng nếu để owr trong phần chính sẽ gây rườm rà. Thông thường các chi tiết được để ở phần phụ lục là kết quả thô(chưa qua xử lý), mã nguồn, phần mềm, thư viện, thông số kỹ thuật, chi tiết linh kiện, hình ảnh minh họa thêm,...
\subsubsection{Tài liệu tham khảo}
\paragraph{Cách liệt kê} \mbox{}

áp dụng cách liệt kê theo quy định của IEEE. theo đó, tài liệu tham khảo được đánh số thứ tự trong ngoặc vuông. thứ tự liệt kê là thứ tự xuất hiện của tài liệu tham khảo được trích dẫn. tài liệu tham khao đã liệt kê bắt buộc phải được trích dẫn trong phần nội dung của đồ án. tài liệu tham khảo cần có nguồn gốc rõ ràng và phải từ nguồn tin cậy. hạn chế trích dẫn tài liệu tham khảo từ các website, wikimedia
\paragraph{Các loại tài liệu tham khảo}\mbox{}

các nguồn tài liệu tham khảo chính là sách, bài báo trong các tạp chí, bài báo trong các hội nghị và các tài liệu tham khảo khác trên internet. 

\subsubsection{Đánh số phương trình}
phương trình được đánh số theo số của chương. như hình sẽ và bảng biểu (phương trình trong chương 1 là 1.1, 1.2,...

\subsubsection{Đánh số định nghĩa, định lý, kết quả}
các định nghĩa, định lý, kết quả được đánh số theo số của chương và được sử dụng chung một chỉ số. ví dụ trong chương 3, các định nghĩa, định lý, hệ quả sẽ được đánh số theo thứ tự như sau: định lý 3.1, định nghĩa 3.2, hệ quả 3.3,...

\newpage
\section*{CHƯƠNG 3. THUẬT TOÁN}
\phantomsection
\addcontentsline{toc}{section}{\numberline{} CHƯƠNG 3. THUẬT TOÁN}
\setcounter{section}{3}
\setcounter{subsection}{0}
\setcounter{figure}{0}
\setcounter{table}{0}
Đây là phần sinh viên tự phát triển như:
\begin{itemize}
    \item xây dựng thuật toán
    \item xây dựng chương trình
    \item mô phỏng, tính toán, thiết kế, chạy thử kết quả
\end{itemize}
\subsection{Cách chèn ẳnh}
\begin{figure}[H]
    \centering
    \includegraphics[width=10cm, height= 10cm]{images/bk_logo.png}
    \caption[Logo Hust]{\fontsize{12pt}{0pt}\selectfont \bfseries{Logo của Đại học Bách Khoa Hà Nội}}
    \label{hình31}
\end{figure}
Hình \ref{hình31} là ví dụ về cách chèn ảnh. lưu ý chú thích của hình vẽ được đặt ngay dưới hình vẽ. tất cả các hình vẽ phải được đề cập đến trong phần nội dung và phải được phân tích và bình luận giống như đang làm :3

\subsection{Cách tạo bảng}
\begin{table}[H]
    \centering
    \caption{Kết quả thí nghiệm}
    \label{bang31}
    \begin{tabularx}{0.85\textwidth}{|>{\centering\arraybackslash}s|>{\centering\arraybackslash}a|>{\centering\arraybackslash}a|>{\centering\arraybackslash}s|}
        \hline
        Lần thí nghiệm   & điện áp đo được  & điện áp tham chiếu  & sai lệch(\%) \\
        \hline
        1 &  &  &\\
        \hline
        2 &  &  &\\
        \hline
        3 &  &  &\\
        \hline
    \end{tabularx}
\end{table}
bảng \ref{bang31} là ví dụ về cách tạo bảng. lưu ý chú thích của bảng được đặt ở ngay trước của bảng. tất cả các bảng biểu phải được đề cập đến trong phần nội dung và phải được phân tích và bình luận giống như nè

\subsection{Cách viết phương trình}
\begin{equation} \label{pt31}
    F(x) = \int^a_b \frac{1}{3}x^3
\end{equation}
Phương trình \ref{pt31} là một phương trình tích phân, là ví dụ về 1 phương tình tích phân. 

Thử phương trình khác
\begin{equation} \label{pt32}
    F(x) = \frac{a}{b}
\end{equation}
Phương trình \ref{pt32} là một phân số dạng thông thường.

\subsection{Cách viết định nghĩa, hẹ quả,...}
Định lý Py-ta-go được áp dụng trong tam giác vuông.
\begin{theorem}\label{dlptg} % định lý
Định lý Py-ta-go: trong một tam giác vuông ta luôn có bình phương của cạnh huyền sẽ bằng tổng bình phương của hai cạnh góc vuông.
\end{theorem}
định lý \ref{dlptg} được áp dụng trong tam giác vuông

\begin{corollary}\label{corol}
blabla

\end{corollary}
Hệ quả \ref{corol} blabla

\begin{lemma}\label{lemma}
hihi bổ đề nha
\end{lemma}
bổ đề \ref{lemma} hihi

\begin{defn}\label{defn}
nội dung :3
\end{defn}
định nghĩa \ref{defn} :3
\newpage

\section*{CHƯƠNG 4. THÍ NGHIỆM VÀ KẾT QUẢ}
\phantomsection
\addcontentsline{toc}{section}{\numberline{} CHƯƠNG 4. THÍ NGHIỆM VÀ KẾT QUẢ}
\setcounter{section}{4}
\setcounter{subsection}{0}
\setcounter{figure}{0}
\setcounter{table}{0}
\lipsum[] 
\begin{figure}[H]
    \centering
    \includegraphics[width=10cm, height=12cm]{images/bk_logo.png}
    \label{Hinh41}
    \caption{Hinh Bách khoa}
\end{figure}
Hình \ref{Hinh41} hihi
\cleardoublepage
\end{document}